\section{Session 02}

\begin{question}{
Which of the following command line \textbf{compile} the program \textit{HelloWorld}:
}
  \false java HelloWorld.java
  \true  javac HelloWorld.java
  \false javac HelloWorld.class
  \false javac
\end{question}

\begin{question}{
Which of the following command line \textbf{execute} the program {HelloWorld}:
}
  \false java HelloWorld.java
  \false java HelloWord.class
  \true java HelloWorld
  \false javac HelloWorld.class
\end{question}

\begin{question}{
Which of the following command display the files and directory of the current directory:
}
  \false pwd
  \false cd
  \false man
  \true ls
\end{question}

\begin{question}{
What is the purpose of comment in code ?
}
  \false better formatting, to make the file easy to ready
  \false to disable part of the code, when no longer needed
  \true to provide information on how the associated code works
  \false to pass metadata to the code
\end{question}

\begin{question}{
How to add copyright and legal stuff on a piece of code
}
  \false use special part of the file that won't be compile
  \false you can't, you have to add a file called COPYRIGHT next to the source code
  \true as a comment, on top of the file
  \false source code cannot be copyright
\end{question}

\begin{question}{
What is a variable ? What does it do ?
}
  \false a variable is way to do operation
  \false a variable is used to store number in memory
  \true a variable allocated enough memory to store a typed information
  \false a variable is an unsigned int
\end{question}

\begin{question}{
Which of the following datatype does \textbf{not} store an integer value:
}
  \false int
  \false short
  \false byte
  \true char
\end{question}

\begin{question}{
Which of the following datatype \textbf{does} store its information on 2 bits:
}
  \false short
  \false byte
  \true boolean
  \false char
\end{question}

\begin{question}{
Which of the following operatir is \textbf{not} aplicable to the data type \texttt{char}:
}
  \false no operator can be applied to char variables
  \false lower than equal (<)
  \true addition (+)
  \false equality (==)
\end{question}

\begin{question}{
Which of the following operator is \textbf{applicable} to the data type \texttt{char}:
}
  \false substraction (-)
  \false addition (+)
  \false division (/)
  \true greater than equal (>)
\end{question}

\begin{question}{
What is an \textbf{array} ?
}
  \false a set of values of different types
  \false a variable containing a reference to one value
  \false a variable holding a reference to the first item of a set of values of different data types
  \true a variable holding a reference to the first item of a set of values of the same data types
\end{question}
