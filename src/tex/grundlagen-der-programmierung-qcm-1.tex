\documentclass[a4paper,10pt]{article}
\usepackage{beamerarticle}
\usepackage[T1]{fontenc}
\usepackage[french]{babel}
\usepackage[utf8]{inputenc}
\usepackage{listings}
\usepackage{color}
\usepackage[pdftex,colorlinks=true,urlcolor=blue]{hyperref}
\usepackage{graphics}
\usepackage{graphicx}
\usepackage{ifthen}
\usepackage{fullpage}
\usepackage{times}
\usepackage{fancybox}
\usepackage[french]{qcm}

\renewcommand{\normalfont}{\sffamily} % Pour avoir de "l'arial"


\begin{document}

\begin{center}
  \begin{Large}
    TODO
  \end{Large}
\end{center}

\section{A - Rappel}

\begin{question}{À l'aide d'une application spécifique, développée de manière "native" (en C++)
et fonctionnant uniquement sur Windows, un agent des ministères peut afficher et modifier son
calendrier. Lorsqu'il a fini, et qu'il appuye sur le bouton "Sauvegarder", les informations contenus
dans le calendrier sont envoyés à un système central et publiées pour être mis à disposition de ces
collaborateurs. Le système central, lui, ne fonctionne pas sur Windows.}
  \true C'est une application de type client/serveur.
  \false C'est une application de type client léger, utilisant les protocoles internet pour inter
  agir avec le serveur.
  \false C'est une application de type client/serveur utilisant (probablement) RMI pour communiquer
  avec le système central.
  \false C'est une application de type client/serveur utilisant (probablement) un Web Service pour
  communiquer avec le serveur central.
\end{question}

\begin{question}{Vous concevez une application destinée à travailler sur un vaste (plusieurs
\textit{giga}) ensemble de données hétéroclite. L'objetif de votre application est de fournir à ses
utilisateurs un maximum de fonctionnalités de recherche sur ces données, permettant, entre autre, de
aisément croiser les informations et les relier entre elles. En première approximation, quelle mode
de persistance, parmi les modes suivants, utiliseriez vous ?}
  \false une simple arboresence de fichiers plats (plusieurs fichiers de petites tailles, réparties
  dans plusieurs répertoires)
  \false un unique fichier XML
  \true une base de donnée relationnelle
  \true une base de donnée non relationnelle (de type "NoSQL" ou un LDAP)
\end{question}


\begin{question}{Au coeur des technologies du \textit{web}, on trouve le protocole HTTP. Quelles
sont les caractéristiques de ce dernier ? \textit{Attention, on parle ici uniquement du protocole
HTTP tel qu'il est spécifié et non des éventuels ajouts/contournement qui ont pu s'y greffer par la
suite...}}
  \false HTTP est un protocole connecté et sans état. On ne peut donc conserver d'état sur le
  serveur mais on peut donc exécuter plusieurs requêtes, au sein d'un même appel HTTP.
  \false HTTP est un protocole asynchrone, qui peut fonctionner même si le serveur à qu'il s'adresse
  est indisponible.
  \true HTTP est un protocole synchrone, mais déconnecté - une fois la réponse à la requête reçue,
  la connection est fermée. En outre, HTTP est sans état, aucun état n'est conservé sur le serveur.
  \true HTTP est un protocole synchrone, sans état et déconnecté - une fois la réponse à la requête
  reçue, la connection est fermée. On peut stocker des informations sur le serveur à l'aide de la
  session.
\end{question}

\section{B - Services applicatifs}

\begin{question}{Pour la réalisation d'un projet interne, vous disposez d'un serveur Apache,
installé et configuré avec le \textit{mod\_php5}, sur une machine \textit{Red Hat Enterprise Linux}.
Vous disposez donc, du point de vue conceptuel:}
  \false d'un moyen de réaliser des applications natives, et non des applications web. Vos
  développeurs devront par ailleurs prendre soin de libérer, de manière explicite, la mémoire qu'ils
  alloueront, et, enfin, l'application réalisée ne sera pas portable, et ne fonctionnera donc que
  sur Linux.
  \false d'un conteneur d'exécution conçu pour réaliser des applications Web. Ce dernier, en outre,
  apporte une certaine portabilité, mais vos développeurs devront prendre soin de libérer, de manière explicite,
  toute allocation mémoire que leur code effectuera.
  \true d'un conteneur d'exécution conçu pour réaliser des applications Web, et prenant en charge le
  nettoyage de la mémoire, sans appel explicite par les développeurs. La stratégie de nettoyage de
  la mémoire est simple, après chaque requête, la mémoire allouée est libéré, aucune données ou
  instructions chargées en mémoire n'est conservées.
  \true d'un conteneur d'exécution conçu pour réaliser des applications Web, et prenant en charge le
  nettoyage de la mémoire, sans appel explicite par les développeurs. La stratégie de nettoyage de
  la mémoire utilise un \textit{Garbage Collector} qui détermine, à interval régulier, si la mémoire
  allouée peut être liberée ou non.
\end{question}

\begin{question}{Pour décrire les propriétés d'une transaction, on utilise l'acronyme ACID. Laquelle
des définitions ci dessous de chacune des lettres de l'acronyme est erronée:}
  \true \textbf{atomique}: la suite d'opérations est indivisible, en cas d'échec en cours
  d'une des opérations, la suite d'opérations doit être complètement annulée (rollback) quel que
  soit le nombre d'opérations déjà réussies.
  \true \textbf{cohérente} le contenu de la base de données à la fin de la transaction doit être
  cohérent sans pour autant que chaque opération durant la transaction donne un contenu cohérent.
  Un contenu final incohérent doit entraîner l'échec et l'annulation de toutes opérations de la
  transaction.
  \false \textbf{isolée} lorsque deux transactions A et B sont exécutées en même temps, elles se
  déroulent sur des ressources différentes, comme par exemple deux base de données distinctes.
  \true \textbf{durable} Une fois validé, l'état de la base de données doit être permanent, et aucun
  incident technique (exemple: crash) ne doit pouvoir engendrer une annulation des opérations
  effectuées durant la transaction.
\end{question}

\begin{question}{Laquelle des assertions ci desous, toute au sujet des caches, est correcte:}
  \false Si l'on change la valeur d'une adresse IP associée à un nom DNS, cette nouvelle valeur est
  propagée immédiatement sur le réseau car il n'y aucune mise en cache de la résolution de nom DNS.
  \false Lorsque l'on met un cache applicatif en place, on est limité par la taille mémoire alouée
  à l'application. % demi point
  \false Un cache améliore beaucoup les performances des applications exécutant beaucoup
  d'opérations d'écriture.
  \true Lorsque l'on met en place un cache, l'objectif est d'avoir un très bon taux de \textit{hit}
  (ce qui signifie que l'application trouve très souvent l'information recherchée dans le cache).
\end{question}

\begin{question}{Pour améliorer les performances de votre application, vous avez mis en place un
cache applicatif, au sein même de cette dernière. Malheureusement, après mesure, celui ci ne peut
contenir que les 10 résultats les plus souvent sollicités par vos utilisateurs et vous avez
déterminé que votre application, pour être plus rapide, doit pouvoir accéder plus rapidement au 100
premiers résultats les plus sollicités par vos utilisateur. Laquelle des stratégies suivantes est
correcte et permettra (vraisemblablement) à votre application d'être plus rapide:}
  \true On place un cache de second niveau, entre l'application et la base de données, par exemple
  une instance MemCached, dans laquelle on place les 100 premiers résultats.
  \false On change la politique d'éviction du cache pour qu'il enlève plus rapidement les éléments
  périmés qu'il contient.
  \false On démarre plus d'instance de l'application pour améliorer les performances globales du
  système.
  \false Il faut modifier l'application pour changer la politique de mise en cache des informations.
\end{question}

\section{C - Intégration et application distribuée}

\begin{question}{Vous devez réaliser une application A qui doit communiquer avec une autre
application B lors de l'un de ses traitements. Elle doit en effet transférer vers B un ensemble de
données, pour que B les traite. Dans quel cas suivant (1 seul) pourriez vous utiliser un MOM pour
implémenter la communication entre A et B:}
  \false Juste après l'envoi des données, l'application A doit utiliser un identifiant, retourné par
  l'application B pour continuer son traitement.
  \false Le service B est implémenté sous forme de Web Service, et votre application (A) doit se
  connecter directement à ce dernier (sans passer par exemple, par un ESB).
  \true Les données envoyées par A à B sont des données statistiques. La fonction du service B est
  essentiellement de stocker et regrouper les données statistiques envoyées par les applications.
  \false Pour communiquer avec le service B, vous devez utiliser RMI.
\end{question}

\begin{question}{Pour accéder à une ressource, on met souvent en place un \textit{connection pool}.
Laquelle des assertions à leurs sujets, qui suivent, n'est pas exacte:}
  \false l'utilisation de \textit{pool} de connexion évite aux applications de baisser en
  performance en lui fournissant une connexion "prête à l'emploi"
  \false l'utilisation d'un \textit{pool} de connexion permet d'assurer qu'un serveur d'application
  abritant plusieurs applications, utilisant toute la même source de données, ne dépassera en aucun
  cas un certain nombre de connexion.
  \true la mise en place d'un \textit{pool} de connexion est complexe, et nécessite de modifier
  l'application. Il n'est pas possible de disposer d'un tel mécanisme en passant par une API
  standard.
  \false On ne peut pas utiliser un \textit{pool} de connexion avec un LDAP.
\end{question}

\begin{question}{Votre application doit offrir deux services accessibles par d'autres programmes. Le
premier (WS A) service permet d'envoyer un ensemble de données cohérents, qui seront ajoutés à la base de
données de votre service - à condition que les données fournies soient valides. L'autre service (WS
B) est un service de supervision, qui permet de savoir dans quel état est votre application. Si il est
invoqué, il retourne simplement le nombre de transaction en cours et une chaine de caractère "OK" ou
"KO" selon si votre application est fonctionnelle ou non. Parmi les choix suivants quel est le plus
adapté pour implémenter votre application:}
  \true SOAP/XML pour WS A et ReST pour le WS B.
  \false ReST pour les deux.
  \false SOAP/XML pour les deux.
  \false SOAP/XML pour le WS B et ReST pour le WS A.
\end{question}

\section{D - Mise à échelle et production}

\begin{question}{Une application a été mise en production avec trois instances, placées derrière un
\textit{load balancer} (Apache, avec le mod\_proxy). L'application répond en moyenne en 3s et les
utilisateurs sont satisfait du temps de réponse. Néanmoins, ils se plaignent que, de temps en temps,
quand une instance tombe, pour une raison ou une autre, ils perdent le travail en cours et doivent
recommencer depuis le début leurs saisies. Quelle stratégie adoptée pour assurer cette reprise sur
panne ?}
  \false Remplacer mod\_proxy par mod\_jk.
  \false Placer un ALTHEON (\textit{load balancer hardware} devant le système.
  \false Mettre en place un \textit{firewall} :)
  \true Mettre en \textit{cluster} l'application
\end{question}

\begin{question}{Pour assurer la surveillance d'une simple application "web", réalisée en Java et
déployée au sein d'un serveur JBoss, quelles indicateurs choissieriez vous ?}
  \false l'utilisation du CPU de la machine, la température du processeur et la \textit{swap}
  \false l'utilisation mémoire de la machine et la taille maximum du \textit{pool} de connexion à la
  base de donnée
  \true le nombre de processus dédiée aux requêtes HTTP, le nombre de session, la consommation de la
  mémoire de la machine virtuelle Java, et le nombre de connexion active à la base de données
  \false la consommation mémoire de la machine virtuelle, la fréquence de passage du \textit{Garbage
  Collector} et sa durée.
\end{question}

\end{document}
